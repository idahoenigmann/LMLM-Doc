% english word for 'eidestattliche erklärung?'

\addcontentsline{toc}{section}{Eidestattliche Erklärung}
\section*{Eidestattliche Erklärung}

\vspace{10mm}

\normalsize
Hiermit erkläre ich an Eides statt, dass ich die vorliegende Arbeit selbstständig und ohne fremde Hilfe verfasst und keine anderen als die im Literaturverzeichnis angegeben Quellen und Hilfsmittel verwendet habe. Insbesondere versichere ich, dass ich alle wörtlichen und sinngemäßen Übernahmen aus anderen Werken als solche kenntlich gemacht habe.

\vspace{1cm}

Wiener Neustadt am \today \\

\vspace{1cm}

{\bf Verfasser / Verfasserinnen:} \\

\vspace{2cm}

  \begin{tabular}{p{.4\textwidth}p{.4\textwidth}}
    \declauthors
  \end{tabular}
