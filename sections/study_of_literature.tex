\chapter{Study of Literature}

\textbf{Author: } 

\section{Different Approaches to the Problem}
[Input: Paper (gleiches Problem ohne NN) finden - Peter]

[Input: Video (eine Kamera, Entfernung zu Punkt (größe bekannt)) finden (ohne NN) - Peter]


[Vielleicht ist irgendetwas davon spannend:
Links im Tex file
%https://ieeexplore.ieee.org/abstract/document/6399589
%https://ieeexplore.ieee.org/abstract/document/100062
%https://ieeexplore.ieee.org/abstract/document/6079296
%https://www.sciencedirect.com/science/article/pii/S0957417414008161
%http://neuralnetworksanddeeplearning.com/chap1.html
%https://patents.google.com/patent/US8164628B2/en
%https://apps.dtic.mil/docs/citations/ADA366182
%https://ieeexplore.ieee.org/abstract/document/1087003
%https://digital-library.theiet.org/content/conferences/10.1049/cp.2010.0495
%https://www.isca-speech.org/archive/interspeech_2010/i10_1045.html
%https://ieeexplore.ieee.org/abstract/document/6639344
%https://www.sciencedirect.com/science/article/pii/B9780127412528500108
%https://en.wikipedia.org/wiki/Visual_cliff#The_study_in_different_species
%https://science.sciencemag.org/content/145/3634/835
%https://en.wikipedia.org/wiki/Visual_cliff
%https://en.wikipedia.org/wiki/Depth_perception
]

\subsection{LIDAR}

\subsection{Structure from Motion}

\subsection{Feature Tracking}

\section{Depth perception}
[TODO: humans, two eyes - gleich wie bei unserem Aufbau]

\subsection{Depth sensation}
[TODO: Pigeons, deer, children (visual cliff)]

\filbreak