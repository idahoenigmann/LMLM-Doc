\chapter{Kurzfassung}

Das Thema dieser Diplomarbeit ist die Distanzbestimmung mit Hilfe eines neuronalen Netzwerks. In vorherigen Arbeiten hatten die Autoren die Aufgabe, die Entfernung zu Objekten in der Umgebung einer Drohne zu erkennen. Dafür konnten sie eine Tiefenkamera verwenden. Allerdings ist es nicht möglich, die Tiefenkamera auf der Drohne zu befestigen. Die Idee dieser Arbeit ist es, die Tiefenkamera wegzulassen und nur die Kamera auf der Drohne zu verwenden. Um das zu verwirklichen simulieren die Autoren die menschlichen Augen, indem sie zwei Bilder vom selben Objekt nehmen, die aus leicht unterschiedlichen Positionen stammen, und daraus die Distanz bestimmen, ähnlich dem menschlichen Gehirn.

Um das neuronale Netzwerk zu trainieren verwendeten die Autoren Blender, ein Programm, das Modellierung von dreidimensionalen Szenen ermöglicht. So modellierten die Authoren Szenen, die unterschiedliche Objekte beinhalten. Mit Hilfe eines Python Skripts wurden dann wiederholt zwei Kameras in der Szene platziert und aus deren Sicht die zwei Bilder des Objektes gerendert. So konnten die Autoren ausreichend Testdaten in der limitierten Zeit generieren, um das neuronale Netzwerk zu trainieren.

Die Autoren verwenden TensorFlow, um das Neuronale Netzwerk zu verwirklichen. Allerdings entschieden sie sich aber auch dazu, eine Eigenimplementierung in C++ zu schreiben um die Vorteile von TensorFlow deutlich zu machen. Diese Eigenimplementierung konnte sich in einfachen Anwendungsfällen zwar gegen Tensorflow behaupten oder teilweise sogar TensorFlow übertreffen, aber für komplexe Aufgaben, wie diese Diplomarbeit, war sie nicht ausreichend entwickelt. Zusammengefasst ist Tensorflow besser getestet und funktioniert auch für komplexe Anwendungsfälle verlässlich gut.

Im Laufe dieser Arbeit wurden zwei Experimente durchgeführt, die das Generalisieren von neuronalen Netzwerken in diesem Anwendungsgebiet untersuchen. Das erste Experiment beschäftigt sich mit der Generalisierung zu unterschiedlichen Objekten. Das Zweite mit der Möglichkeit nach einem Lernvorgang mit ausschließlich computergenerierten Bildern, Bilder, die von einer realen Kamera stammen, zu verarbeiten.

Diese Diplomarbeit zeigt, dass es sehr schwer ist, Wirklichkeit mit computergenerierten Bildern nachzustellen. Die zurückgelieferten Distanzen waren für computergenerierte Bilder akzeptabel, aber das neuronale Netzwerk, welches nur mit computergenerierten Bildern trainiert wurde, hatte Schwierigkeiten mit realen Bildern.


\chapter{Abstract}

This diploma thesis deals with distance estimation realised via a neural network. In previous work the authors were challenged to detect objects in the surrounding of a drone, which they realised with a depth camera. The problem of this approach was the difficulty of mounting the camera on the drone, which proved impossible. The general idea of this work is therefore to omit the depth camera and only use the pre-existing camera of the drone. The authors decide to simulate the human eyes by taking two pictures of the same object from slightly different positions. The distance to some object can then be determined from these two images, similarly a human brain.

In order to train the neural network the authors used Blender, a program which allows users to model three dimensional scenes. The authors used this functionality to model scenes containing multiple different objects. With the help of a Python script two cameras were repeatedly placed in the scene and images of the objects from two different points of view were created. This way the authors could generate enough training data to train the neural network in the limited time frame.

The authors used TensorFlow to implement the neural network. Additionally an implementation of a neural network was written in C++ from scratch, which shows the advantages of using TensorFlow. The neural network written by the authors proved successful and even outperformed TensorFlow in simple use cases, for more complex ones, as this diploma thesis, the C++ implementation was not developed enough. Summarized TensorFlow is well tested and works very reliably in complex use cases.

During this work the authors performed two experiments, testing if the neural network is able to generalise in the described use case. The first experiment deals with generalising to other objects. The second one explores the possibility of the neural network generalising after having been trained on computer-generated images only, to process images taken by a real camera.

The results of this diploma thesis are, that it is very difficult to represent reality with computer-generated images. The retrieved distances were acceptable, as long as computer-generated images were used. But the neural network, trained exclusively with computer-generated images, had problems with real images.