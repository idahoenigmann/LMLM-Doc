\chapter{Kurzfassung}

\vspace{10mm}

Das Thema dieser Diplomarbeit ist die Distanzbestimmung mit Hilfe eines neuronalen Netzwerks. Die Autoren hatten die Idee zu dieser Diplomarbeit, weil sie früher eine eigene Kamera mit Tiefenerkennung verwenden mussten, als sie mit Drohnen arbeiteten. Die Idee ist, diese Tiefenkamera wegzulassen um nur die Kamera auf der Drohne zu verwenden. Um das zu verwirklichen simulieren die Autoren die menschlichen Augen, indem sie zwei Bilder vom selben Objekt nehmen, die aus leicht unterschiedlichen Positionen stammen, und daraus die Distanz bestimmen, ähnlich dem menschlichen Gehirn.\\
Um das neuranale Netzwerk zu trainieren verwendeten die Autoren Blender, ein Programm zur Modellierung, um Szenen zu modellieren, die Objekte beinhalten. Mit Hilfe eines Skripts wurden dann wiederholt zwei Kameras in der Szene plaziert und aus deren Sicht die zwei Bilder des Objektes gerendert. So konnten die Autoren ausreichend Testdaten in der verfügbaren Zeit generieren, um das neuronale Netzwerk zu trainieren.\\
Die Autoren verwendeten Tensorflow, um das Neuronale Netzwerk zu verwirklichen, entschieden sich aber auch dazu, eine Eigenimplementierung in C++ zu schreiben, um die Vorteile von Tensorflow deutlich zu machen. Diese Eigenimplementierung konnte sich in simplen Anwendungsfällen zwar gegen Tensorflow behaupten oder teilweise sogar Tensorflow übertreffen, aber für komplexe Anwendungsfälle, wie diese Diplomarbeit, war sie nicht ausreichend Entwickelt und wichtige Features für diese Anwendungsfälle fehlten. Weil die Eigenimplementierung nicht das Hauptthema dieser Diplomarbeit ist, entschieden sich die Autoren dafür, diese nicht weiterzuentwickeln. Zusammengefasst ist Tensorflow viel besser getestet und funktioniert auch für komplexe Anwendungsfälle verlässlich gut.\\
Die Ergebnisse des neuronalen Netzwerks in Tensorflow und dieser Diplomarbeit sind, das es sehr schwer ist Wirklichkeit mit computergenerierten Bildern nachzustellen. Die zurückgelieferten Distanzen waren für computergenerierte Bilder akzeptabel, aber das neuronale Netzwerk, welches selbst mit computergenerierten Bildern trainiert wurde, hatte Schwierigkeiten mit realen Bildern.


\chapter{Abstract}

This diploma thesis is about distance estimation with the help of a neural network. The authors had this idea, because they needed an external camera with depth perception to determine distances before and now they want to realise this with a single camera - present on a drone - which had no depth perception, in order to omit the external camera. Since a single camera is used, the authors decided to simulate the human eyes by taking two pictures of the same object from slightly different positions and determine the distance that way, to replicate the work of the human brain.\\
In order to train the neural network the authors used blender, a modelling program, to model scenes in which objects are placed. With the help of a script two cameras were repeatedly placed in the scene, which took the two pictures from the object. This way the authors could generate enough training data to train the neural network in the given time frame.\\
The neural network was implemented with tensorflow, but the authors decided to test tensorflow and write an implementation of neural networks on their own in C++, in order to show the gains of using tensorflow. The author's implementation could hold their own against, or even outperform, tensorflow on simple use cases, but for complex use cases like this diploma thesis the author's implementation was just not enough developed, missing features important for these complex use cases. Because the C++ implementation was not the main topic of this diploma thesis, the authors did not develop it further. So overall tensorflow is way better tested and guaranteed to work well even with complex use cases.\\
The results of the neural network in tensorflow and of this diploma thesis are, that it is very difficult to represent reality with computer-generated images. The retrieved distances were acceptable, as long as computer-generated images were used. But the neural network, trained with computer-generated images, had problems with real images.