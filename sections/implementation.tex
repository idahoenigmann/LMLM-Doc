\chapter{Implementation}

\textbf{Author: } 

\section{Generating test data}
Good test data is one of the most important things in machine learning. The system knows only what is depicted in the training data, which is why it is important to include as many aspects of the problem as possible in this data.

Since machine learning needs a lot of data in order to solve the given task it can be tiresome to generate and label all this data by hand. Therefore we decided to simulate some objects using a computer graphics software called Blender.

Blender allows for relatively easy generation of training data by providing a Python API.

[TODO: Image camera setup, lightning, objects in Blender]
[TODO: Renders and labels for example objects]

\section{OpenCV}

[TODO: was ist das? wofür wird es allgemein verwendet? wofür verwenden wir es?]
[TODO: Codebeispiele?]

\section{Neural Network}

[TODO: Erklärung - zuerst Tensorflow, dann selbst implementiert in C++]

\subsection{Tensorflow}

[TODO: was ist Tenserflow, welche Sprachen werden unterstützt, wie wird es (sonst noch) verwendet?]

\subsection{Alternatives to Tensorflow}

[+ warum verwenden wir ausgerechnet Tenserflow]

\subsection{Structure of our Neural Network}

[wie lesen wir Daten ein, wie viele layer, was ist der output (maximale entfernung? z.B. 10m)]

\subsection{C++ Implementation}

\subsection{Technical difficulties}

\filbreak