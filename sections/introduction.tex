\chapter{Introduction}

\textbf{Author: Ida Hönigmann}

\vspace{2mm}

Robots are getting more and more mobile. While a few years ago their usage was mostly limited to aid factory automation, robots have found widespread adoption in a multitude of industries, such as self driving cars and autonomous delivery drones.
A challenge frequently encountered is navigating in unknown environments, which either requires the robot to sense specific characteristics of its surroundings or to communicate with some external system.

The problem of navigation has been looked at from many different angles. One popular approach in mobile robotics is to use GPS, an external positioning system. In order to determine the position of a robot using GPS it has to establish communication with at least four satellites. The exact position of each satellite as well as the current time is broadcasted by the satellites. By measuring the time needed for the signal to reach the robot the position can be calculated up to three meters accuracy.
However, in some cases positioning a robot using external positioning methods is no possible. In the case of GPS this can be due to obstacles interfering with the radio signals send by the satellites.
In comparison, we focus on a system that can navigate outdoor as well as indoor. 

\section{Goal}
The goal of this diploma thesis is to implement a system which can localize a robot using no other sensors than a camera. This limitation was purposely chosen as our system will be used by future robotic students at our school and many robot systems used in the field of education are only poorly equipped with sensors that are able to detect its environment. One sensor, that many robots used for educational purposes are either already equipped with, or that can be easily mounted is a camera.

As part of our work we not only want to implement an easy to use API for future robotic students, but to also show the possibilities and advantages of using machine learning for this challenge.

In order to accomplish precise localization in various different surroundings, we plan on implementing a neural network. The neural network should take images, taken by the camera as an input, and output the relative distance of any object shown in the images. By using machine learning we hope to be less dependent on a specific situation or setup in comparison to different camera based localisation methods. For example the localization should work on objects varying in size and shape, as well as in different situations of lighting.

\filbreak
