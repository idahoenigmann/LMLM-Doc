\chapter{Introduction}

\textbf{Author:}

Robots are getting more and more mobile. While a few years ago their usage was mostly limited to aid atomization in factories, they now can be found as self driving cars or autonomous delivery drones.
One of the main challenges all mobile robots must face is navigation. To overcome this challenge it often is necessary to sense the environment, in order to accomplish a robust working navigation towards a given target.

The problem of navigation has been looked at from many different angles. One popular approach in mobile robotics when navigating outdoor is to use GPS, an external positioning system. [TODO: erklärung GPS]

In our work, however, we focus on a system that has different use cases. A typical use case for our system is a drone which should land next to some target object. If the absolute position of the object is not known to the needed precision GPS can not be used.
Landing next to this object can be done by retrieving the relative position of the robot to the object from some sensor mounted on the robot.
Another difference to GPS is that we want our system to work outdoor as well as indoor.

\section{Goal}
The goal of this diploma thesis is to show the possibilities and advantages of using machine learning for localization, and to provide an API for easy use by future robotics students. In order to do localization a neural network will evaluate two images with different points of view. From these two points of view it will return the relative distance from the point of view of the second image to any given object visible in both images. Machine Learning will be used in order to not be dependent on a specific situation or setup. The localization should hereby work on, for example, objects varying in size and/or in different situations of lighting.

\textbf{[TODO: länger, und nicht das werden wir machen, sondern das ist unser Ziel]}

\filbreak
